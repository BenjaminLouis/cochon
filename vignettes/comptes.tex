\documentclass[]{article}
\usepackage{lmodern}
\usepackage{amssymb,amsmath}
\usepackage{ifxetex,ifluatex}
\usepackage{fixltx2e} % provides \textsubscript
\ifnum 0\ifxetex 1\fi\ifluatex 1\fi=0 % if pdftex
  \usepackage[T1]{fontenc}
  \usepackage[utf8]{inputenc}
\else % if luatex or xelatex
  \ifxetex
    \usepackage{mathspec}
  \else
    \usepackage{fontspec}
  \fi
  \defaultfontfeatures{Ligatures=TeX,Scale=MatchLowercase}
\fi
% use upquote if available, for straight quotes in verbatim environments
\IfFileExists{upquote.sty}{\usepackage{upquote}}{}
% use microtype if available
\IfFileExists{microtype.sty}{%
\usepackage{microtype}
\UseMicrotypeSet[protrusion]{basicmath} % disable protrusion for tt fonts
}{}
\usepackage[margin=1in]{geometry}
\usepackage{hyperref}
\hypersetup{unicode=true,
            pdftitle={Compte-rendu du weekend cochon du 1er au 3 février 2019},
            pdfauthor={Benjamin Louis},
            pdfborder={0 0 0},
            breaklinks=true}
\urlstyle{same}  % don't use monospace font for urls
\usepackage{graphicx,grffile}
\makeatletter
\def\maxwidth{\ifdim\Gin@nat@width>\linewidth\linewidth\else\Gin@nat@width\fi}
\def\maxheight{\ifdim\Gin@nat@height>\textheight\textheight\else\Gin@nat@height\fi}
\makeatother
% Scale images if necessary, so that they will not overflow the page
% margins by default, and it is still possible to overwrite the defaults
% using explicit options in \includegraphics[width, height, ...]{}
\setkeys{Gin}{width=\maxwidth,height=\maxheight,keepaspectratio}
\IfFileExists{parskip.sty}{%
\usepackage{parskip}
}{% else
\setlength{\parindent}{0pt}
\setlength{\parskip}{6pt plus 2pt minus 1pt}
}
\setlength{\emergencystretch}{3em}  % prevent overfull lines
\providecommand{\tightlist}{%
  \setlength{\itemsep}{0pt}\setlength{\parskip}{0pt}}
\setcounter{secnumdepth}{5}
% Redefines (sub)paragraphs to behave more like sections
\ifx\paragraph\undefined\else
\let\oldparagraph\paragraph
\renewcommand{\paragraph}[1]{\oldparagraph{#1}\mbox{}}
\fi
\ifx\subparagraph\undefined\else
\let\oldsubparagraph\subparagraph
\renewcommand{\subparagraph}[1]{\oldsubparagraph{#1}\mbox{}}
\fi

%%% Use protect on footnotes to avoid problems with footnotes in titles
\let\rmarkdownfootnote\footnote%
\def\footnote{\protect\rmarkdownfootnote}

%%% Change title format to be more compact
\usepackage{titling}

% Create subtitle command for use in maketitle
\providecommand{\subtitle}[1]{
  \posttitle{
    \begin{center}\large#1\end{center}
    }
}

\setlength{\droptitle}{-2em}

  \title{Compte-rendu du weekend cochon du 1er au 3 février 2019}
    \pretitle{\vspace{\droptitle}\centering\huge}
  \posttitle{\par}
    \author{Benjamin Louis}
    \preauthor{\centering\large\emph}
  \postauthor{\par}
      \predate{\centering\large\emph}
  \postdate{\par}
    \date{26-02-2019}


\begin{document}
\maketitle

{
\setcounter{tocdepth}{2}
\tableofcontents
}
\hypertarget{bilan-du-weekend}{%
\section{Bilan du weekend}\label{bilan-du-weekend}}

\hypertarget{deroulement-du-weekend}{%
\subsection{Déroulement du weekend}\label{deroulement-du-weekend}}

--------------- A VENIR ----------------------

\hypertarget{ce-qui-a-ete-dit}{%
\subsection{Ce qui a été dit}\label{ce-qui-a-ete-dit}}

Suite à ce merveilleux weekend, quelques phrases de bilan ont été
évoquées. Les voici donc :

\begin{quote}
\begin{itemize}
\tightlist
\item
  Très bon week-end (on s'en fout je sais, c'est pas très constructif)
\item
  trop de sel dans la terrine et pas assez dans le jambonneau (qui, au
  passage, a été validé par Olivier, il a dit essai réussi)
\item
  Peut-être mettre un peu moins de matière dans les moules à jambonneau
  pour avoir plus de gelée. Faire cuire avec de l'eau salée et un
  bouquet garni. Possibilité de faire 2 fournées de jambonneaux si on
  veut
\item
  prévoir plus de verdure si on veut plus de caillettes (répondre aux
  appels à l'aide d'Estelle le moment venu)
\item
  Faire plus cuire les boudins blancs si possible
\item
  Les boudins créoles plus épicés (et pourtant je suis sensible)
\item
  Ne pas oublier d'acheter les noisettes en plus pour le saucisson (on
  peut partir par exemple sur 80g /kg, c'est ce qu'on a eu fait)
\item
  pour le pâté de tête, j'en ai pas pris mais je suis curieux de
  connaître votre avis. Possibilité d'utiliser les moules d'Olivier pour
  en faire du frai (un peu comme les jambonneaux)
\item
  Très bon weekend et supers repas nombreux, j'ai bcp aimé ça.
\item
  Même remarque que Luc sur la terrine au four, moins de sel ; la
  quantité de noisettes est super ;et idem pour les boudins blancs, ça
  pourrait être au moins 20'.
\item
  Le pâté de tête est bien bon, il fait presque rillettes, l'intérêt
  c'est que ça a inspiré des amis qui n'aiment pas d'habitude. C'est
  bien les poireaux dedans. Peu salé mais salé comme il faut je trouve.
\item
  Le dimanche d'avant, on pourrait juste s'avancer en lavant des pots et
  des cagettes plastique si besoin (parce que pendant le weekend, je
  trouve qu'on en a marre d'avoir les mains mouillées)
\item
  on est super content de vous avoir tous accueilli pour ce week-end
\item
  merci à ``ceux qui savent'' quoi faire quand et comment, moi ça me va
  très bien de me laisser porter, sans le soucis ! (Marian)
\item
  trop bonnes les rillettes, parfaites à notre gout
\item
  ok pour le sel dans la terrine et dans les jambonneaux (trop et pas
  assez) et les épices dans le créole
\item
  pour le reste on n'a pas encore tout gouté !
\item
  l'année prochaine, on proposera de faire un boudin ardéchois (il
  restait du sang !)
\item
  allez les verts (pour les caillettes)
\end{itemize}
\end{quote}

\hypertarget{generalites}{%
\section{Généralités}\label{generalites}}

\hypertarget{quelques-chiffres-sur-les-cochons}{%
\subsection{Quelques chiffres sur les
cochons}\label{quelques-chiffres-sur-les-cochons}}

Un peu de données générales sur les cochons. Dans ce tableau je
distingue la viande (morceaux qu'on garde entier) de la chair (morceaux
hachés). Les poids de viande et de chair donnent un total d'environ
133kg ce qui donne un rendement de 57\%. Dans ce rendement, il y a
cependant un peu d'os (jambon, côte de porc), donc il est plutôt estimé
à la hausse.

Tableau 1

Nombre de cochons

Poids total (kg)

Prix au kilo (€/kg)

Prix total (€)

Poids de viande (kg)

Poids de chair (kg)

Poids des foies (kg)

Poids de gras (kg)

Rendement sans foie (\%)

Rendement avec foie (\%)

Poids viandes + chair + foies (kg)

Prix viandes + chair + foies au kilo (€/kg)

2

232.6

4.9

1139.74

85.14

48.17

3.4

32.6

57

59

136.71

8.34

\hypertarget{quelques-chiffres-sur-les-consommables-utilises}{%
\subsection{Quelques chiffres sur les consommables
utilisés}\label{quelques-chiffres-sur-les-consommables-utilises}}

---------------- A VENIR ----------------------

\hypertarget{prix-global-de-la-viande}{%
\subsection{Prix global de la viande}\label{prix-global-de-la-viande}}

Pour calculer un prix pour les différentes recettes, j'ai ramené le prix
de la carcasse au poids de viande + chair + foies. Le foie étant
limitant, j'ai choisi de lui donner une valeur au même titre que la
viande (sinon le pâté de foie ne valait rien). Au final, on obtient un
prix de 8,34 €/kg.

\hypertarget{repartition-dans-les-recettes}{%
\subsection{Répartition dans les
recettes}\label{repartition-dans-les-recettes}}

Le prochain tableau donne les poids de viande, chair, foies et total
qu'il y a dans chaque recette et l'équivalent en prix. Ce tableau me
sert surtout pour calculer ci-après le poids unitaire des recettes en
fonction de leur conditionnement.

Tableau 2

recettes

poids\_viandes

poids\_foie

poids\_chair

poids\_total

prix\_total

pate\_foie

1.0

1.00

8.34

pate\_campagne

1.4

1.40

2.80

23.34

pate\_tete

rillettes

jambonneau

2.50

2.50

20.84

terrine\_four

1.0

2.00

3.00

25.01

caillettes

3.80

3.80

31.68

bocaux\_viandes

11.00

11.00

91.71

bocaux\_jambonneau

0.98

0.98

8.17

boudin\_charentais

boudin\_creole

boudin\_blanc

3.00

3.00

25.01

saucisses\_fraiches

8.00

8.00

66.69

chipolatas

8.97

8.97

74.78

saucissons

14.00

14.00

116.72

saucisses\_seches

5.00

5.00

41.68

chorizo

2.00

2.00

16.67

cote\_de\_porc

8.36

8.36

69.70

rouelle

7.25

7.25

60.44

roti

14.10

14.10

117.55

filet\_mignon

2.70

2.70

22.51

ribs

5.80

5.80

48.35

poitrine

5.30

5.30

44.19

jambon

24.65

24.65

205.50

autres\_viandes\_seches

2.50

2.50

20.84

\hypertarget{prix-des-differentes-recettes-en-fonction-de-leur-conditionnement}{%
\section{Prix des différentes recettes en fonction de leur
conditionnement}\label{prix-des-differentes-recettes-en-fonction-de-leur-conditionnement}}

\hypertarget{methode-de-calcul}{%
\subsection{Méthode de calcul}\label{methode-de-calcul}}

Bien qu'un travail collectif de collecte de données ait été effectué
lors de ce week-end, certaines données sont manquantes ou trop
imprécises pour pouvoir calculer un prix exact de chaque recette.
Certaines simplifications méthodologiques ont du être faites.

\begin{itemize}
\item
  Le prix d'une recette est calculé par rapport au prix de la quantité
  de viande et/ou foie et/ou chair qu'elle contient
\item
  Le gras n'étant pas limitant il n'est pas comptabilisé (ou son prix
  est égal à 0€ si vous préférez). Ainsi, plus une recette contient en
  proportion du gras, moins elle est chère en unité de poids. He oui, le
  cholestérol ne vaut pas grand chose!
\item
  Les autres ingrédients des différentes recettes sont intégrés dans le
  prix total des courses du weekend qui est divisé au prorata du nombre
  de personnes. Le prix des ingrédients n'est donc pas contenu dans le
  prix des recettes.
\item
  De même, les boyaux sont intégrés au prix total des courses et non pas
  dans le prix des différentes recettes. On pourrait en théorie le faire
  mais les données sont trop imprécises pour que ça ait du sens. Une
  estimation très rapide \emph{``à la louche''} conduit à un prix
  d'environ 30 cents de boyau pour une saucisse sèche ou un saucisson,
  environ 20 cents de boyau pour une chipo et environ 3 cents de boyau
  pour une saucisse fraiche ou un boudin.
\end{itemize}

\textbf{Méthode 1}

Lorsque les recettes sont conditionnées en produit de différentes
tailles, un premier calcul consiste à évaluer le prix unitaire des
différents conditionnement. La taille du conditionnement donne un
coefficient qui, associé à la quantité faite de chaque conditionnement
et le prix total de la recette, permet de etrouver le prix unitaire. Je
sais que beaucoup s'en foute, mais moi j'adore mettre des formules,
alors la voici :

\[PUproduit_{ij} = \frac{coef_{ij} \times P_{total_{j}}}{\sum_{i}\left(coef_{ij} \times quantite_{ij}\right)}\]

où \(PUproduit_{ij}\) est le prix unitaire du produit (ou
conditionnement) \(i\) d'une recette \(j\), \(coef_{ij}\) est le
coefficient associé à ce conditionnement, \(P_{total_{j}}\) est le prix
de revient total de la recette \(j\) (cf Tableau 2) et \(quantite_{ij}\)
est la quantité totale faite de contionnement \(i\) pour la même recette
\(j\).

Une fois qu'on a le prix unitaire d'un conditonnement, il suffit de le
multiplier par la quantité que chaque personne/famille a pris pour
savoir ce que doit la personne/famille pour cette recette :

\[Prix_{jk} = \sum_{i}\left(nb_{ijk} \times PUproduit_{ij}\right)\] où
\(Prix_{jk}\) est le prix que doit payer la personne/famille \(k\) pour
la recette \(j\), \(nb_{ijk}\) est le nombre du conditionnement \(i\) de
la recette \(j\) prix par la personne/famille \(k\) et
\(PUproduit_{ij}\) est le prix unitaire du produit (ou conditionnement)
\(i\) de la recette \(j\).

\textbf{Méthode 2}

Si il n'y a qu'un conditionnement pour une recette, c'est directement la
quantité prise par une personne/famille qui est utilisée comme
coefficient associé au prix total de la recette pour calculer ce que
doit la personne/famille. Ce coefficient peut être un nombre d'unité
prise (e.g.~j'ai pris 10 caillettes), un poids (e.g.~j'ai pris 500g de
chipolatas) ou une part (e.g.~j'ai pris une part de jambon).

\[Prix_{jk} = \frac{w_{jk} \times P_{total_{j}}}{\sum_{k} w_{jk}}\] où
où \(Prix_{jk}\) est le prix que doit payer la personne/famille \(k\)
pour la recette \(j\), \(w_{jk}\) est le coefficient représentant la
quantité de la recette \(j\) prise par la personne/famille \(k\) et
\(P_{total_{j}}\) est le prix de revient total de la recette \(j\)
(Tableau 2).

Pour les plus passionnés, suspicieux, fous ou ceux qui s'ennuient,
normalement vous avez tout dans ce document pour refaire les calculs et
vérifier que je ne me suis pas trompé. Mais attention, les chiffres qui
sont dans les tableaux peuvent avoir des décalages de quelques grammes
ou centimes à cause des arrondis (j'allais pas laisser plus de deux
chiffres décimaux). par contre, vous verrez qu'à la fin on tombe bien!

\hypertarget{terrine-et-pates}{%
\subsection{Terrine et pâtés}\label{terrine-et-pates}}

\hypertarget{pate-de-foie}{%
\subsubsection{Pâté de foie}\label{pate-de-foie}}

La méthode 1 a été appliquée pour le pâté de foie donnant un prix
unitaire de 0.46 € pour le spots de 130 ml et de 0.93 € pour les pots de
260 ml (oui c'est pas tout à fait le double mais souvenez-vous, les
arrondis~!!!).

Tableau 3

produit

quantite

coef

prix\_unitaire

pate\_foie\_130ml

2

130

0.46

pate\_foie\_260ml

8

260

0.93

Ce qui en fonction de ce que chacun a pris donne le tableau (Tableau 4)
suivant des prix à payer par personne/famille pour cette recette.

Tableau 4

qui

doit\_payer

anaelle\_matthieu

0.00

casse

0.00

commun

0.00

dons

0.46

estelle\_marian

1.85

laura\_thomas

0.93

marion\_judicael

0.93

steph\_luc

0.93

suk\_ben

1.39

tiffany

1.85

\hypertarget{pate-de-campagne}{%
\subsubsection{Pâté de campagne}\label{pate-de-campagne}}

Même travail pour le pâté de campagne acvec la méthode 1.

Tableau 5

produit

quantite

coef

prix\_unitaire

pate\_campagne\_130ml

5

130

0.52

pate\_campagne\_200ml

8

200

0.79

pate\_campagne\_260ml

14

260

1.03

Et le tableau des prix à payer.

Tableau 6

qui

doit\_payer

anaelle\_matthieu

3.17

casse

1.03

commun

0.00

dons

0.52

estelle\_marian

3.88

laura\_thomas

3.65

marion\_judicael

2.58

steph\_luc

2.34

suk\_ben

3.09

tiffany

3.09

\hypertarget{terrine-au-four}{%
\subsubsection{Terrine au four}\label{terrine-au-four}}

La terrine au four a été divisée en 8 parts (dont une donnée). La
méthode 2 est utilisée.

Tableau 7

qui

doit\_payer

anaelle\_matthieu

3.13

casse

0.00

commun

0.00

dons

3.13

estelle\_marian

3.13

laura\_thomas

3.13

marion\_judicael

3.13

steph\_luc

3.13

suk\_ben

3.13

tiffany

3.13

\hypertarget{jambonneau}{%
\subsubsection{Jambonneau}\label{jambonneau}}

Même méthode 2 pour les jambonneaux qui ont été divisés en 8 parts dont
une consommée en commun.

Tableau 8

qui

doit\_payer

anaelle\_matthieu

2.61

casse

0.00

commun

2.61

dons

0.00

estelle\_marian

2.61

laura\_thomas

2.61

marion\_judicael

2.61

steph\_luc

2.61

suk\_ben

2.61

tiffany

2.61

\hypertarget{caillettes}{%
\subsubsection{Caillettes}\label{caillettes}}

Les caillettes sont considérées comme ayant toutes le même poids de
viande. Le nombre prit par chaque personne/famille fait office de
coefficient pour la méthode 2.

Tableau 9

qui

doit\_payer

anaelle\_matthieu

2.26

casse

0.00

commun

0.00

dons

6.79

estelle\_marian

5.66

laura\_thomas

2.26

marion\_judicael

3.39

steph\_luc

3.39

suk\_ben

4.53

tiffany

3.39

\hypertarget{pate-de-tete-et-rillettes}{%
\subsubsection{Pâté de tête et
rillettes}\label{pate-de-tete-et-rillettes}}

D'un commun accord, le pâté de tête et les rillettes sont considérées
comme ayant un coût de viande égal à 0€ car issus de partie qui serait
jetées sinon.

\hypertarget{boudin}{%
\subsection{Boudin}\label{boudin}}

\hypertarget{boudin-blanc}{%
\subsubsection{Boudin blanc}\label{boudin-blanc}}

Le boudin blanc contient de la chair et on considère que chaque boudon
en contien la même quantité. On utilise la méthode 2 avec comme
coefficients le nombre de boudin pris par chaque personne/famille.

Tableau 10

qui

doit\_payer

anaelle\_matthieu

2.99

casse

0.00

commun

0.75

dons

1.49

estelle\_marian

5.97

laura\_thomas

0.75

marion\_judicael

2.24

steph\_luc

4.48

suk\_ben

3.73

tiffany

2.61

\hypertarget{boudins-noir-et-creole}{%
\subsubsection{Boudins noir et créole}\label{boudins-noir-et-creole}}

D'un commun accord, les boudins noir et créole sont considérés comme
ayant un coût de viande égal à 0€ car issus de partie qui serait jetées
sinon.

\hypertarget{saucisses-fraiches-et-chipolatas}{%
\subsection{Saucisses fraiches et
chipolatas}\label{saucisses-fraiches-et-chipolatas}}

\hypertarget{saucisses-fraiches}{%
\subsubsection{Saucisses fraiches}\label{saucisses-fraiches}}

Pour les saucisses fraiches, on considère que chaque saucisse contient
la même quantité de viande. Sur le coût de revient total de cette
recette (Tableau 2), des saucisses ont été prises seules et d'autres
sont dans certains bocaux de viande (4 dans les bocaux de saucisses
seules et 2 dans le sbocaux de saucisses/viande). Il faut d'abord
calculer le prix de revient d'une saucisses en considérant la totalité
faite de saucisses. En comptant les fraiches et celles mises en bocaux,
au total, 78 saucisses ont été faites. Le tableau 2 nous dit que le coût
total des saucisses est de 66.69€ donc le coût unitaire d'une saucisse
est de 0.86€.

Le tableau 11 donne ce que doit payer chaque personne/famille sur les
saucisses fraiches seulement avec la méthode 2 en prenant comme
coefficient le nombre de saucisses fraiches prises. Celles mises en
bocaux seront comptabilisées dans le prix des bocaux.

Tableau 11

qui

doit\_payer

anaelle\_matthieu

5.13

casse

0.00

commun

11.12

dons

0.00

estelle\_marian

0.00

laura\_thomas

2.57

marion\_judicael

0.00

steph\_luc

0.00

suk\_ben

5.13

tiffany

0.00

\hypertarget{chipolatas}{%
\subsubsection{chipolatas}\label{chipolatas}}

La méthode 2 est également utilisée ici avec comme coefficients le poids
de chipolatas pris par chaque personne/famille, ce qui en fait surement
un coût plus précis que pour les saucisses fraiches.

Tableau 12

qui

doit\_payer

anaelle\_matthieu

14.76

casse

0.00

commun

0.00

dons

2.58

estelle\_marian

11.67

laura\_thomas

12.76

marion\_judicael

7.59

steph\_luc

6.00

suk\_ben

11.67

tiffany

7.75

\hypertarget{saucisses-seches-et-saucissons}{%
\subsection{Saucisses sèches et
saucissons}\label{saucisses-seches-et-saucissons}}

\hypertarget{saucisses-seches}{%
\subsubsection{Saucisses sèches}\label{saucisses-seches}}

On considère ici aussi que chaque saucisse sèche contient la même
quantité de viande et le nombre prix par chaque personne/famille sert de
coefficient dans la méthode 2.

\textbf{N.B.} à mon avis ici il y a plus de différence que pour les
saucisses fraiches où les caillettes. Il serait surement plus précis les
autres années de peser avant séchage chaque saucisse est de répartir
directement. Cela demande un peu plus de logistique car il faut marquer
chaque saucisse pour redistribuer convenablement après séchage.

Tableau 13

qui

doit\_payer

anaelle\_matthieu

4.17

casse

0.00

commun

0.00

dons

0.00

estelle\_marian

8.34

laura\_thomas

4.17

marion\_judicael

8.34

steph\_luc

4.17

suk\_ben

8.34

tiffany

4.17

\hypertarget{saucissons}{%
\subsubsection{Saucissons}\label{saucissons}}

Même travail que pour les saucisses sèches avec la même remarque sur une
pesée et répartition avant séchage.

Tableau 14

qui

doit\_payer

anaelle\_matthieu

17.29

casse

0.00

commun

0.00

dons

0.00

estelle\_marian

17.29

laura\_thomas

17.29

marion\_judicael

12.97

steph\_luc

17.29

suk\_ben

17.29

tiffany

17.29

\hypertarget{chorizo}{%
\subsubsection{Chorizo}\label{chorizo}}

Même travail et remarque que pour les saucisses sèches et saucissons.

Tableau 15

qui

doit\_payer

anaelle\_matthieu

4.17

casse

0.00

commun

0.00

dons

0.00

estelle\_marian

2.08

laura\_thomas

2.08

marion\_judicael

2.08

steph\_luc

2.08

suk\_ben

2.08

tiffany

2.08

\hypertarget{bocaux-viandes}{%
\subsection{Bocaux viandes}\label{bocaux-viandes}}

\hypertarget{bocaux-viandes-etou-saucisses}{%
\subsubsection{Bocaux viandes et/ou
saucisses}\label{bocaux-viandes-etou-saucisses}}

Les bocaux de viande sont conditionnés en bocaux de différentes tailles
avec de la viande seule, de la saucisse seule ou bien les deux. La
méthode 1 est appliquée avec la petite différence qu'il faut ajouter le
prix des saucisses lorsqu'il y en a. Pour les coefficients liés à a
viande, on considère que le bocaux de viande pour 1-2 personnes
contiennent environ 200g, ceux de 3-4 personnes environ 400g, ceux qui
mélangent viande et saucisses environ 300g. Pour les saucisses, on
considère qu'il y en a 2 dans le bocaux saucisses et viande et 4 dans
les bocaux de saucisses uniquement.

Tableau 16

produit

quantite

coef\_viande

qtite\_saucisse

prix\_viande

prix\_saucisse

prix\_unitaire

bocaux\_saucisses

7

0

4

0.00

3.42

3.42

saucisses\_viandes

11

300

2

2.57

1.71

4.28

viandes\_1-2p

11

200

0

1.71

0.00

1.71

viandes\_3-4p

13

400

0

3.43

0.00

3.43

Le bocal de saucisses seules revient à 3,42€, le bocal de mélangent
viande saucisses à 4,28€, le bocal de viande pour 1-2 personnes à 1,71€
et le bocal de viande pour 3-4 personnes à 3,43€.

Tableau 17

qui

doit\_payer

anaelle\_matthieu

6.85

casse

6.86

commun

0.00

dons

0.00

estelle\_marian

30.83

laura\_thomas

34.26

marion\_judicael

6.85

steph\_luc

10.28

suk\_ben

23.98

tiffany

14.56

\hypertarget{bocaux-jambonneau}{%
\subsubsection{Bocaux jambonneau}\label{bocaux-jambonneau}}

On considère que la même quantité de jambonneau a été mise dans les 3
bocaux.

Tableau 18

qui

doit\_payer

anaelle\_matthieu

0.00

casse

0.00

commun

0.00

dons

0.00

estelle\_marian

2.72

laura\_thomas

0.00

marion\_judicael

2.72

steph\_luc

0.00

suk\_ben

0.00

tiffany

2.72

\hypertarget{viandes-fraiches}{%
\subsection{Viandes fraiches}\label{viandes-fraiches}}

\hypertarget{cote-de-porc}{%
\subsubsection{Côte de porc}\label{cote-de-porc}}

La méthode 2 est appliquée avec comme coeffficients les poids pris par
chacun.

Tableau 19

qui

doit\_payer

anaelle\_matthieu

19.09

casse

0.00

commun

0.00

dons

4.25

estelle\_marian

6.25

laura\_thomas

12.09

marion\_judicael

11.34

steph\_luc

0.00

suk\_ben

16.68

tiffany

0.00

\hypertarget{rouelle}{%
\subsubsection{Rouelle}\label{rouelle}}

La méthode 2 est appliquée avec comme coeffficients les poids pris par
chacun.

Tableau 20

qui

doit\_payer

anaelle\_matthieu

14.76

casse

0.00

commun

0.00

dons

0.00

estelle\_marian

12.09

laura\_thomas

9.17

marion\_judicael

10.75

steph\_luc

0.00

suk\_ben

13.67

tiffany

0.00

\hypertarget{roti}{%
\subsubsection{Roti}\label{roti}}

La méthode 2 est appliquée avec comme coeffficients les poids pris par
chacun.

Tableau 21

qui

doit\_payer

anaelle\_matthieu

28.60

casse

0.00

commun

0.00

dons

5.92

estelle\_marian

15.84

laura\_thomas

12.76

marion\_judicael

16.59

steph\_luc

7.00

suk\_ben

22.84

tiffany

8.00

\hypertarget{filet-mignon}{%
\subsubsection{Filet mignon}\label{filet-mignon}}

Le filet mignon a été mangé ensemble, son prix total va dans les
dépenses communes.

Tableau 22

qui

doit\_payer

anaelle\_matthieu

0.00

casse

0.00

commun

22.51

dons

0.00

estelle\_marian

0.00

laura\_thomas

0.00

marion\_judicael

0.00

steph\_luc

0.00

suk\_ben

0.00

tiffany

0.00

\hypertarget{ribs}{%
\subsubsection{Ribs}\label{ribs}}

Les ribs seront mangées ensemble, leur prix total va dans les dépenses
communes.

Tableau 23

qui

doit\_payer

anaelle\_matthieu

0.00

casse

0.00

commun

48.35

dons

0.00

estelle\_marian

0.00

laura\_thomas

0.00

marion\_judicael

0.00

steph\_luc

0.00

suk\_ben

0.00

tiffany

0.00

\hypertarget{viandes-seches}{%
\subsection{Viandes sèches}\label{viandes-seches}}

\hypertarget{poitrine}{%
\subsubsection{Poitrine}\label{poitrine}}

On considère que les huit poitrines ont le même poids (car elles n'ont
pas été pesées individuellement avant pesage). Le nombre de poitrines
prises par personne/famille correspond aux coefficients pour la méthode
2.

Tableau 24

qui

doit\_payer

anaelle\_matthieu

11.05

casse

0.00

commun

0.00

dons

0.00

estelle\_marian

11.05

laura\_thomas

11.05

marion\_judicael

0.00

steph\_luc

0.00

suk\_ben

5.52

tiffany

5.52

\hypertarget{jambon}{%
\subsubsection{Jambon}\label{jambon}}

On divisera le jambon en 7 parts égales que l'on répartira entre chaque
personne/famille.

Tableau 25

qui

doit\_payer

anaelle\_matthieu

29.36

casse

0.00

commun

0.00

dons

0.00

estelle\_marian

29.36

laura\_thomas

29.36

marion\_judicael

29.36

steph\_luc

29.36

suk\_ben

29.36

tiffany

29.36

\hypertarget{autres-viandes}{%
\subsubsection{Autres viandes}\label{autres-viandes}}

Comme pour le jambon, on divisera le jambon en 7 parts égales que l'on
répartira entre chaque personne/famille.

Tableau 26

qui

doit\_payer

anaelle\_matthieu

2.98

casse

0.00

commun

0.00

dons

0.00

estelle\_marian

2.98

laura\_thomas

2.98

marion\_judicael

2.98

steph\_luc

2.98

suk\_ben

2.98

tiffany

2.98

\hypertarget{prix-a-payer-par-personne}{%
\section{Prix à payer par personne}\label{prix-a-payer-par-personne}}

\hypertarget{la-viande}{%
\subsection{La viande}\label{la-viande}}

Une fois qu'on a fait la calcul pour toutes les recettes, pour connaître
ce que doit la personne/famille \(k\), il suffit de sommer l'ensemble de
ce qu'elle doit pour chaque recette.

\[Prix_{k} = \sum_{i} Prix_{jk}\] où \(Prix_{k}\) est le prix que doit
la personne/famille \(k\) pour l'ensemble des recettes.

Tableau 27

qui

doit\_payer

anaelle\_matthieu

172.34

casse

7.89

commun

85.33

dons

25.14

estelle\_marian

173.60

laura\_thomas

163.83

marion\_judicael

126.43

steph\_luc

96.03

suk\_ben

178.02

tiffany

111.12

\hypertarget{achats-commun}{%
\subsection{Achats commun}\label{achats-commun}}

Le tableau 28 résume les achats communs, toutes personnes/familles
confondues. Sont ajoutés dedans le coût de revient de la viande en
commun i.e.~celle mangée ensemble (\texttt{commun}), celle donnée
(\texttt{dons}) et la casse qu'il y a eu (\texttt{casse}).

Tableau 28

achat

doit\_payer

bocaux

90.00

boyaux\_boeuf

20.37

boyaux\_chipolatas

46.40

boyaux\_porc

3.08

boyaux\_saucisses

12.20

courses

178.63

joints\_bocaux

20.60

legumes

60.00

sel

15.60

viennoiserie

20.00

cochon\_casse

7.89

cochon\_commun

85.33

cochon\_dons

25.14

Le coût de ces achats communs sont répartis entre les personnes/familles
au prorata du nombre de personnes, les enfants n'étant pas comptés.

Tableau 29

qui

doit\_payer

anaelle\_matthieu

90.04

estelle\_marian

90.04

laura\_thomas

90.04

marion\_judicael

90.04

steph\_luc

90.04

suk\_ben

90.04

tiffany

45.02

\hypertarget{total}{%
\subsection{Total}\label{total}}

Pour savoir ce que doit payer chaque personne/famille, on somme le coût
de la viande pris par chacun et le coût des dépenses communes que doit
chacun pour obtenir le tableau 29.

Tableau 30

qui

doit\_payer

anaelle\_matthieu

262.38

estelle\_marian

263.63

laura\_thomas

253.87

marion\_judicael

216.47

steph\_luc

186.07

suk\_ben

268.06

tiffany

156.14

\hypertarget{qui-doit-a-qui}{%
\section{Qui doit à qui ?}\label{qui-doit-a-qui}}

Le tableau 31 fait un bilan de ce que chacun a payé, ce que chacun doit
payer et la différence entre ce que chacun doit payer et ce que chacun a
payé. En gros, si c'est positif, c'est la somme que vous devez, si c'est
négatif, c'est la somme qu'on vous doit. Donc tout le monde doit de
l'argent à Steph\&Luc, à Clément et au Mazel.

Je propose ici que tout le monde donne ce qu'il doit à Steph\&Luc et que
ces derniers remboursent Clément et le Mazel.

Tableau 31

qui

a\_paye

doit\_payer

difference

anaelle\_matthieu

110.94

262.38

151.44

clement

28.73

0.00

-28.73

estelle\_marian

80.00

263.63

183.63

mazel

74.95

0.00

-74.95

steph\_luc

1273.04

186.07

-1086.97

suk\_ben

29.33

268.06

238.73

tiffany

9.63

156.14

146.51

laura\_thomas

0.00

253.87

253.87

marion\_judicael

0.00

216.47

216.47

\hypertarget{annexe}{%
\section{Annexe}\label{annexe}}

Tableau des répartitions des recettes entre les personnes/familles

recettes

produit

steph\_luc

estelle\_marian

anaelle\_matthieu

laura\_thomas

marion\_judicael

tiffany

suk\_ben

commun

dons

casse

pate\_foie

pate\_foie\_260ml

1

2

0

1

1

2

1

0

0

0

pate\_foie

pate\_foie\_130ml

0

0

0

0

0

0

1

0

1

0

pate\_campagne

pate\_campagne\_260ml

1

3

0

2

2

2

3

0

0

1

pate\_campagne

pate\_campagne\_200ml

1

1

4

2

0

0

0

0

0

0

pate\_campagne

pate\_campagne\_130ml

1

0

0

0

1

2

0

0

1

0

pate\_tete

pate\_tete\_260ml

0

1

0

0

1

3

2

0

0

0

pate\_tete

pate\_tete\_130ml

0

1

1

2

1

2

2

0

0

0

rillettes

rillettes\_500ml

0

0

0

1

0

0

0

0

0

0

rillettes

rillettes\_260ml

2

4

5

4

3

2

6

1

0

0

rillettes

rillettes\_130ml

3

2

2

1

2

3

0

1

0

1

rillettes

rillettes\_350ml

0

1

0

0

0

1

1

0

0

0

jambonneau

jambonneau

1

1

1

1

1

1

1

1

0

0

terrine\_four

terrine\_four

1

1

1

1

1

1

1

0

1

0

caillettes

caillettes

6

10

4

4

6

6

8

0

12

0

bocaux\_viandes

viandes\_3-4p

1

3

1

2

1

1

2

0

0

2

bocaux\_viandes

viandes\_1-2p

2

0

0

4

0

2

3

0

0

0

bocaux\_viandes

saucisses\_viandes

0

4

0

4

0

1

2

0

0

0

bocaux\_viandes

bocaux\_saucisses

1

1

1

1

1

1

1

0

0

0

bocaux\_jambonneau

bocaux\_jambonneau

0

1

0

0

1

1

0

0

0

0

boudin\_charentais

boudins\_charentais

4

7

8

6

5

6

9

11

11

0

boudin\_charentais

boudins\_charentais\_430ml

0

1

1

0

0

2

0

0

0

0

boudin\_charentais

boudins\_charentais\_250ml

0

0

0

1

0

0

0

0

0

0

boudin\_creole

boudins\_creole

3

4

8

5

5

8

11

11

2

0

boudin\_creole

boudins\_creole\_250ml

0

0

0

0

1

1

1

0

0

0

boudin\_blanc

boudins\_blanc

12

16

8

2

6

7

10

2

4

0

saucisses\_fraiches

saucisses\_fraiches

0

0

6

3

0

0

6

13

0

0

chipolatas

chipolatas

720

1400

1770

1530

910

930

1400

0

310

0

saucissons

saucissons

4

4

4

4

3

4

4

0

0

0

saucisses\_seches

saucisses\_seches

1

2

1

1

2

1

2

0

0

0

chorizo

chorizo

1

1

2

1

1

1

1

0

0

0

cote\_de\_porc

cote\_de\_porc

0

750

2290

1450

1360

0

2001

0

510

0

rouelle

rouelle

0

1450

1770

1100

1290

0

1640

0

0

0

roti

roti

840

1900

3430

1530

1990

960

2740

0

710

0

filet\_mignon

filet\_mignon

0

0

0

0

0

0

0

2700

0

0

ribs

ribs

0

0

0

0

0

0

0

5800

0

0

poitrine

poitrine

0

2

2

2

0

1

1

0

0

0

jambon

jambon

1

1

1

1

1

1

1

0

0

0

autres\_viandes\_seches

autres\_viandes

1

1

1

1

1

1

1

0

0

0


\end{document}
